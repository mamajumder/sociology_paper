%\documentclass{article}
\documentclass[10pt]{article}
\usepackage{times}
%\usepackage{natbib}
%\usepackage{multicol}
\RequirePackage{natbib}
\usepackage{amsmath, amssymb, fullpage, amsthm, array,,graphicx,asa, url}
%\usepackage[dvips]{graphics}

%\usepackage{hyperref} % for hyper reference

\graphicspath{{images/}}

% \usepackage{pifont} % this package is used to print check mark \checkmark
% \linespread{1.6} % factor 1.6 = double space

%\usepackage{setspace}
%\doublespacing



\setlength{\oddsidemargin}{0in}
\setlength{\evensidemargin}{0in}
\setlength{\textwidth}{6.5in}
\setlength{\topmargin}{-0.4in}
\setlength{\textheight}{9in}
\evensidemargin 
\oddsidemargin

\newtheorem{thm}{Theorem}[section]
\newtheorem{dfn}{Definition}[section]
\newtheorem{cor}{Corollary}[thm]
\newtheorem{con}{Conjecture}[thm]
\newtheorem{lemma}[thm]{Lemma}

%\topmargin -0.10in   % when making pdf
%\textheight 9.15in  % when making pdf

\pdfminorversion=4 % as instructed by JASA file upload


\begin{document}

\tableofcontents

% Article top matter
\title{Impact of Demographics and Skills of the Observer on Visual Statistical Inference }
\author{{Mahbubul Majumder, Heike Hofmann, Dianne Cook}
\thanks{Mahbubul Majumder is a PhD student (e-mail: mahbub72@gmail.com) , Heike Hofmann is an Associate  Professor and Dianne Cook is a Professor in the Department of Statistics and Statistical Laboratory, Iowa State University, Ames, IA 50011-1210. This research is supported in part by the National Science Foundation Grant \# DMS 1007697.}}
\date{\vspace{-.5in}}
%\date{\today}  %\today is replaced with the current date
\maketitle

\begin {abstract}  
Visual Inference is dependent on the careful evaluation of the lineups by individual observers. Each individual is different in their cognitive phycology and judiciousness which can affect the power of visual inference. To estimate the power of visual inference this can be controlled by getting evaluations from multiple people from a diverse population. But the other factor that may affect the power of visual inference is the demographics and skills of the observer. In this paper we examine this in details. The simulation experiments suggest that individual skill as well as demographics are very significant for the power of visual inference. Moreover, the data shows the evidence of learning or getting skilled while sequentially evaluating multiple lineups.

{\bf Keywords: \sf statistical graphics, lineup, non-parametric test, cognitive phycology, visualization, exploratory data analysis} 
\end {abstract}

%\begin{multicols}{2}
%\twocolumn

\section{Introduction} 

The fundamental concept of visual statistical inference is introduced by \citet{buja:2009} and later these concepts are validated by \citet{majumder:2012}.

\subsection{Visual Inference} A short introduction of visual inferential procedure.
\subsection{Estimation of Power of Visual Inference} Discussion of the methods for estimating the power of visual inference \citep{majumder:2012}. 

\section{Factors Affecting the Power of Visual Inference} A brief description of the factors that may affect the power of visual inference.

\subsection{Choice of Visual Test Statistic} Visual test statistics as defined in \citep{majumder:2012} serve two main purposes. One is to display a prominent pattern that should direct the human observers to select the actual plot in the lineup when the null hypothesis is not true. The another feature is that it should not show any strange pattern when null hypothesis is true, i.e., it should show similar pattern that null plots may have. Thus a visual test statistic is highly associated with the hypothesis. To achieve these purposes it is very important to decide which plot type and plot features should be adopted. As discussed in \citep{majumder:2012} following grammar of graphics developed  by \citet{wilkinson:1999} and later improved by \citet{hadley:2009} can give the task a standard control. 

Some of the effective features of visual test statistics are discussed in \citep{heike:2012} including plot type, color and shape of the plots. It is also observed that a scatterplot may do a better job than a box plot when using as a visual test statistic for regression parameters \citep{majumder:2012} .  \citep{niladri:2012} presents some \textit{distance measure} to determine how a plot may be different from another.

Plots to do: for the same signal strength show the proportion correct for scatterplot and boxplot. eg., qplot(effect, ump-visual power, color= plot-type)

\subsection{Question that Human Observer Answers} When a observer is prepared to evaluate a lineup, there needs to be a guideline what exactly should be looked for in a lineup. The researcher knows about the hypothesis but the observer does not know the underlying details of the lineup. Thus the researcher needs to ask a question observer needs to answer while evaluating the lineup. This question should provide the observer a little clue so that the answer reflects the hypothesized patterns in the actual plot. For example Table \ref{tbl:visual_stat} shows the questions asked for the simulation experiments done by  \citet{majumder:2012}. Notice that for case 1 if the observer can identify the actual plot in the lineup that should indicate that the plot chosen has the most vertical difference between groups A and B which is exactly what the researchers intend to examine. Similarly for case 2, a correct evaluation would indicate that the slope is different than the slope that may show up just from randomness.


\begin{table*}[hbtp] 
\caption{Visual test statistics and question asked to the observers to answer while evaluating a lineup } 
\centering 
\begin{tabular}{m{0.5cm}m{2cm}m{3cm}m{7.5cm}} 
\hline\hline 
Case & Statistic & Test Statistic & Lineup question \\ [0.5ex] % inserts table %heading 
\hline 
1  & Box plot & \begin{minipage}[t]{3cm} 	\scalebox{0.45}{\includegraphics{stat_category.pdf}} \end{minipage} & Which set of box plots shows biggest vertical difference 
between group A and B? \\
2 &  Scatter plot & \begin{minipage}[t]{3cm}   \scalebox{0.45}{\includegraphics{stat_beta_k.pdf}}\end{minipage} & Of the scatter plots below which one shows data that has steepest slope? \\ 

\hline 
\end{tabular} 
\label{tbl:visual_stat} 
\end{table*} 

These questions are very crucial for the power of visual test. They help observer think in a controlled direction. Notice that there may be unnecessary patterns in the actual data plot which may not necessarily indicate the existence of the significant signal in the plot. These question help observer not to be misguided by those patterns. To review further on this \citet{majumder:2012} have also asked why the observers choose a specific plot. 



\subsection{Observer Personality} Each person is different from others in some way. For example, in a controlled experimental study \citet{zhao:2012}  notice that some people spend a lot of time to decide no matter whether the lineup is difficult or easy while some simply glance at lineups to make a decision. This influences the response of the observer and we see the subject specific variation in the power of visual test \citep{majumder:2012}. 

\subsection{Visual Perception of Human Eye} Different people observe the lineup in different ways.  With the help of eye-tracking equipment \citet{zhao:2012} track the observers' eyes to see how they go through the plots in a lineup to come to their answers. The result suggest that people have particular methods of reading lineups. Some people read lineups from left to write direction while some read from upward to downward. Some people start looking from the center of the lineup while others start from the top left corner. In the earlier phase of the exercise, the observer tend to scan the plots and then start comparing plots to make a final decision. Beside right-left or up-down directions observer show some diagonal movement too. 

Given this pattern of human eye tracks, it may effect the performance of an observer depending on where the actual plot is placed in the lineup. If the actual plot is on the top left corner and the observer starts from that point it may be easier to detect the actual plot earlier in the exercise and the observer could get plenty of time to make comparison. Thus the position of the actual plot in a lineup has some impact.


\subsection{Demographics of Observer}  Age, gender and location of the observer.
\subsection{Skills of Observer} Knowledge about statistical plot as well as education level.

\section{Simulation Experiment for Learning Effect}

\subsection{Experiment Setup} Description of the simulation experiment giving focus on demographics and skills of experimental subjects.
\subsection{Estimation of sample size}  How the sample size is estimated
\subsection{Data collection methods}  Description of the simulation experiment giving focus on demographics and skills of experimental subjects.

\section{Results}
\subsection{Overview of the Data}
\subsection{Diversity of the Experimental Subjects} Amazon Mechanical Turk \cite{turk} web site is a good source for recruiting people to evaluation lineups. It is a source of diverse workers who can give us feedback on visual inspection. This is the kind of task where human are better than computers.  Figure \ref{} shows the location from where we received data. Notice that we received data from around the world and we observed diversity of location in the collected data. We notice diversity in not only location of the observers but also their gender, age groups and education levels. Table \ref{} shows that we have data from almost similar size of male and female population. 

\subsection{Learning Trend of the Observer} Since each subject is shown multiple lineups, they have a chance for self learning and do better on lineups shown later in the sequence. We examine this by fitting generalized mixed effect model presented in \cite{majumder:2012} with covariates attempt number and $p$-value of the lineup. The $p$-value of the lineup measures some sort of difficulty and including this should adjust for the difficulty level of the lineup while estimating the probability of correct responses in each attempt. Figure \ref{fig:learning_trend} shows the fitted subject wise learning pattern as well as overall learning pattern for each of the experiments. The parameter estimates od the model are shown in Table \ref{tbl:model_par}.


%\begin{table}[hbtp]
%\caption{Parameter estimates of mixed model. Estimates are highly significant with $p$-value $<$  0.0001 for all three experiment data.}
%\begin{center}
%\begin{tabular}{cr@{.}lcc}
%  \hline \hline
% &  \multicolumn{3}{c} {Fixed effect}  & Random effect\\
% \cline{2-4}
% Experiment & \multicolumn{2}{l}{Estimate}  &Std. error & Variance\\
%  \hline
%  1 & 0&39 & 0.0094 & 0.0080 \\ 
%  2 & 1&21 &  0.0197 &  0.0443 \\ 
%  3 & 0&59 (Intercept)  &   0.1668 & 1.9917\\ 
%     & 0&21 (Slope)    &  0.0511     &  0.0245\\ 
%     &-0&78 (correlation) & & \\
%   \hline
%\end{tabular}
%\end{center}
%\label{tbl:model_par}
%\end{table}




% latex table generated in R 2.15.0 by xtable 1.7-0 package
% Fri Sep  7 10:28:47 2012
\begin{table}[hbtp]
\caption{Fixed effect parameter estimates of generalized mixed model. Note that attempt is not significant for experiment 2. The continuous covariate, lineup difficulty, was measured by the $p$-value of the actual plot in the lineup.}
\begin{center}
\begin{tabular}{llrrrrl}
  \hline
& Parameters & Estimate & Std..Error & z.value & P-value  & \\ 
  \hline
\multicolumn{2}{l}{\bf{Experiment 1} } &&&& &\\
&(Intercept) & -0.30 & 0.10 & -2.89 & 0.00 & ***\\ 
 & attempt & 0.08 & 0.02 & 4.21 & 0.00 & ***\\ 
 & lineup difficulty & -11.32 & 1.01 & -11.19 & 0.00 & ***\\ 
\multicolumn{2}{l}{\bf{Experiment 2} } &&&&& \\
 & (Intercept) & 2.36 & 0.15 & 16.09 & 0.00 & ***\\ 
 & attempt & 0.01 & 0.03 & 0.35 & 0.72 \\ 
 & lineup difficulty & -55.03 & 2.20 & -25.00 & 0.00 & ***\\ 
\multicolumn{2}{l}{\bf{Experiment 3} } &&&& \\
 & (Intercept) & 0.25 & 0.16 & 1.58 & 0.11 & .\\ 
 & attempt & 0.10 & 0.03 & 3.34 & 0.00 & ***\\ 
 & lineup difficulty & 3.00 & 0.56 & 5.36 & 0.00 & ***\\ 
   \hline
\end{tabular}
\end{center}
\label{tbl:model_par}
\end{table}

% \footnote{Signif. codes: 0 �***� 0.001 �**� 0.01 �*� 0.05 �.� 0.1}


\begin{figure}[htbp] %  figure placement: here, top, bottom, or page
   \centering
   \includegraphics[width=6in]{learning_trend.pdf} 
   \caption{Probability of correct response for different attempts by each subjects for a plot with lineup difficulty ($p$-value of actual plot) = 0.05 . The overall probability increases with attempts indicating a learning trend of the observers for experiments 1 and 3. For experiment 2 the overall trend trend is not statistically significant. Subject to subject variability is seen in learning pattern.}
   \label{fig:learning_trend}
\end{figure}

\section{Conclusion}


\bibliographystyle{asa}
\bibliography{references}

\end{document}



